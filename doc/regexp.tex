\documentclass[a4paper,francais]{article}
\usepackage[T1]{fontenc}
\usepackage[latin1]{inputenc}
\begin{document}

les expressions comprises par Eliot ne sont qu'un sous ensemble des
expressions rationnelles habituelles.

\section{utilisation}

\subsection{caract�res}

\begin{itemize}
\item \texttt{a} � \texttt{z} : le caract�re indiqu�
\item \texttt{.} :  n'importe quel caract�re
\item \texttt{:v:} : n'importe quelle voyelle 
\item \texttt{:c:} : n'importe quelle consonne
\end{itemize}

\subsection{r�p�titions}

\begin{itemize}
\item \texttt{+} une fois ou plus
\item \texttt{*} z�ro fois ou plus
\end{itemize}

\subsection{disjonction}

\begin{itemize}
\item \texttt{[abc]} : \texttt{a} ou \texttt{b} ou \texttt{c}
\end{itemize}

\subsection{groupement}

\begin{itemize}
\item \texttt{\(abc\)} : la cha�ne \texttt{abc}
\end{itemize}

\subsection{exemples}

\end{document}
